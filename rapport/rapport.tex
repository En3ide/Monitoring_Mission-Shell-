\documentclass{article}

\usepackage[utf8]{inputenc}
\usepackage[T1]{fontenc}
\usepackage[french]{babel}

\usepackage{titling}  % Pour personnaliser l'apparence du titre
\usepackage{lipsum}   % Pour du texte d'exemple (optionnel)
\usepackage{setspace} % Pour régler les espacements
\usepackage[colorlinks=true, linkcolor=black]{hyperref}

% Tableau
\usepackage[table]{xcolor}
\usepackage{array}
\usepackage{tabularx} % À inclure dans le préambule pour ajuster la largeur du tableau
\usepackage{xcolor} % Pour colorer les cellules ou le texte si nécessaire

\usepackage{microtype}
\usepackage{enumitem} % Permet de personnaliser les listes

% Diagramme
\usepackage{tikz}
\usetikzlibrary{arrows.meta, positioning, shapes.geometric}

\setlength{\parindent}{0pt}
\hypersetup{linkcolor=red}

\title{\Huge \textbf{Rapport SAE Shell} \\[0.5cm] \Large Monitoring dans la console}
\author{\large Présenté par\\ \textbf{Tim Lamour} \& \textbf{Jamel Bailleul}}
\date{\today}

\begin{document}

\maketitle % Affiche le titre
\tableofcontents % Affiche la table des matières
\newpage

% 0. Membres du groupe
\section{Membres du groupe}
\begin{itemize}
    \item \textbf{Tim Lamour} : développement des fonctions de récupération des informations sur les ressources du système, création et écriture des logs, optimisations, mise en forme du code, documentation et rédaction.
    \item \textbf{Jamel Bailleul} : développement de l'interface graphique et du programme principal.

\end{itemize}

% 1. Contextes
\section{Contexte}
Le projet consiste à développer une \textbf{application de Monitoring}, qui s'exécute dans la console.
Celle-ci permet de surveiller en temps réel les ressources du système, telles que le processeur (CPU), la carte graphique (GPU), la mémoire (RAM), les disques, le réseau, et les processus actifs.
\vspace{1em}

Toutes les données récoltées par l'application sont écrites et mis à jour dans un fichier de sortie appelé le \textbf{fichier de logs}.
\\
Certains aspects graphiques et techniques comme les couleurs ou l'intervalle d'écriture dans le logfile sont personnalisables à l'aide d'un fichier d'entré appelé le \textbf{fichier de configuration}.

% 2. Algorithme, fichier et fonctions
\section{Algorithme, fichiers et fonctions}
Le projet est séparé en 4 fichiers :
\begin{itemize}
    \item \texttt{main.sh} : lit le configfile et lance le programme principale.
    \item \texttt{interface.sh} : contient les fonctions permettant d'afficher et de mettre à jour l'interface.
    \item \texttt{recup\_info.sh} : contient les fonctions permettant de récupérer\\ les informations des ressources du système.
    \item \texttt{update\_log.sh} : contient les fonctions permettant de créer et de mettre à jour le fichier des logs. 
\end{itemize}
\vspace{1em}

\newpage
Voici les fonctions principales par fichiers :
\begin{itemize}
    \item \texttt{main.sh} :
        \begin{itemize}[label=\textbullet]
            \item \texttt{config\_file} : lit le configfile et met à jour les valeurs par défaut de l'interface.
        \end{itemize}
    \vspace{0.3em}
    \item \texttt{interface.sh} :
        \begin{itemize}[label=\textbullet]
            \item {\texttt{clear\_screen}} : permet de nettoyer la console et de dessiner la bordure du terminal.
            \item \texttt{info\_reduite} : permet d'afficher les informations sur les ressources du système une par une si elles sont disponibles.
            \item \texttt{affiche\_processus} : permet d'afficher les informations sur les processus à une position sur le terminal.
            \item \texttt{print\_bar\_h} : permet un pourcentage sous forme d'une barre à une position précise sur le terminal.
            \item \texttt{info\_scinder} : permet de rassembler et scinder tous les affichages précédents.
        \end{itemize}
        \vspace{0.3em}
    \item \texttt{recup\_info.sh} :
        \begin{itemize}[label=\textbullet]
            \item \texttt{recup\_mem} : permet de récupérer les informations sur la mémoire de la machine.
            \item \texttt{recup\_cpu} : permet de récupérer les informations sur la carte graphique de la machine.
            \item \texttt{recup\_gpu} : permet de récupérer les informations sur le processeur de la machine.
            \item \texttt{recup\_disk} : permet de récupérer les informations sur les disques de la machine.
            \item \texttt{recup\_processus} : permet de récupérer les processus en cours d'utilisation de la machine.
            \item \texttt{recup\_network} : permet de récupérer les informations sur l'utilisation réseau de la machine.
        \end{itemize}
        \vspace{0.3em}
    \item \texttt{update\_log.sh} :
        \begin{itemize}[label=\textbullet]
            \item \texttt{create\_logfile} : permet de créer le readconfigfile s'il n'existe pas déjà.
            \item \texttt{write\_in\_logfile} : permet d'écrire dans le readconfigfile et donc de le mettre à jour.
        \end{itemize}
\end{itemize}

\newpage
% 3. Diagramme
\section{Diagramme}

Voici un diagramme montrant la logique d'exécution de l'application :
\vspace{1em}
  
\begin{center}
\begin{tikzpicture}[node distance=1.2cm, >={Stealth[round]}, thick]

    % Styles for different types of nodes
    \tikzstyle{startstop} = [rectangle, rounded corners, minimum width=3.5cm, minimum height=0.8cm, text centered, draw=black, fill=gray!20]
    \tikzstyle{process} = [rectangle, minimum width=3.5cm, minimum height=0.8cm, text centered, draw=black, fill=blue!10]
    \tikzstyle{decision} = [rectangle, minimum width=3.5cm, minimum height=0.8cm, text centered, draw=black, fill=yellow!20]
    \tikzstyle{arrow} = [thick,->,>=stealth]

    % Nodes
    \node (start) [startstop] {Lancement du programme};
    \node (readconfigfile) [decision, below of=start, yshift=-0.5cm] {Lecture du configfile (si fournit)};
    \node (terminal) [process, below of=readconfigfile, yshift=-0.5cm] {Délimitation du terminal};
    \node (resetlogfile) [decision, below of=terminal, yshift=-0.5cm] {Réinitilisation du logfile (si demandé)};
    \node (loop) [startstop, below of=resetlogfile, yshift=-0.5cm] {Boucle (while true)};
    \node (retrieve) [process, below of=loop, yshift=-0.3cm] {Récupérer et afficher les informations};
    \node (update) [process, below of=retrieve, yshift=-0.3cm] {Mise à jour du logfile};
    \node (stop) [startstop, below of=update, yshift=-0.3cm] {Arrêt manuel du processus};

    % Arrows
    \draw [arrow] (start) -- (readconfigfile);
    \draw [arrow] (start.west) -- ++(-1.3,0) |- ([xshift=-0.2cm] terminal.west);
    \draw [arrow] (readconfigfile) -- (terminal);
    \draw [arrow] (terminal) -- (resetlogfile);
    \draw [arrow] (terminal.east) -- ++(1.6,0) |- (loop.east);
    \draw [arrow] (resetlogfile) -- (loop);
    \draw [arrow] (loop) -- (retrieve);
    \draw [arrow] (retrieve) -- (update);
    \draw [arrow] (update.west) -- ++(-1.6,0) |- ([xshift=0.2cm] loop.west);
    \draw [arrow] (update) -- (stop);

\end{tikzpicture}
\end{center}

% 4. Commandes et options de lancement
\newpage
\section{Commandes et options de lancement}

\subsection{Lancer l'application}
Pour exécuter l'application, vous pouvez utiliser la commande suivante dans le terminal :  
\texttt{./main.sh <configfile>}

\subsection{Fichier de configuration}
Le \textbf{configfile} est un fichier de configuration permettant de définir les couleurs de l'interface, l'intervalle pour l'écriture et la réinitialisation du fichier de logs.
  
Il doit respecter la syntaxe suivante : \texttt{nom\_variable=valeur}, avec une affectation par ligne. Si cette syntaxe n'est pas respectée, le processus ne se lance pas.
\vspace{1em}

La majorité des variables doivent obligatoirement avoir une \textbf{couleur} (tableau \ref{tab:couleur}) ou un \textbf{caractère UNICODE} (tableau \ref{tab:unicode}) parmis celles et ceux disponibles.
  
Veuillez vous référer aux tableaux \ref{tab:configable_color_var}, \ref{tab:configable_unicode_var} et \ref{tab:configable_other_var} pour consulter les variables possibles ainsi que leurs valeurs.

% 5. Exemple d'utilisation
\section{Exemple d'utilisation}
Voici un exemple : \texttt{./main.sh configfile.txt}
  
\vspace{0.5em}
Contenu du fichier de configuration \textbf{configfile.txt} :
\begin{verbatim}
bg_color=DARK_BLACK
font_color=DARK_RED
border_color=DARK_BLUE
font_processus_color=BRIGHT_WHITE
\end{verbatim}


% 6. Conclusion
\section{Conclusion}
Dans l'ensemble, nous sommes satisfaits du résultat. Cependant, il reste bien évidemment des possibilités d'optimisation, notamment au niveau de l'aspect graphique, qui est pas le plus esthétique,
et de la manière dont les informations sont récupérées et mises à jour. On pourrait par exemple utilisé du multi-processing.

% 7. Annexe
\newpage
\section{Annexe : tableaux des variables configurables, des couleurs et caractères unicodes disponibles}
\vspace{3em}
\subsection{Variables configurables avec une couleur :}

\begin{table}[h!]
    \centering
    \renewcommand{\arraystretch}{1.5}
    \footnotesize
    \begin{tabular}{|>{\centering\arraybackslash}m{3.5cm}|>{\centering\arraybackslash}m{3.8cm}|>{\centering\arraybackslash}m{3cm}|}
        \hline
        \textbf{Nom de la variable} & \textbf{Signification} & \textbf{Valeur par défaut (couleur)} \\
        \hline
        bg\_color & Couleur de fond de l'interface & DARK\_BLACK \\
        \hline
        font\_color & Couleur de la police principale & DARK\_RED \\
        \hline
        border\_color & Couleur de la bordure & DARK\_BLUE \\
        \hline
        font\_processus\_color & Couleur de la police pour les processus affichés & BRIGHT\_WHITE \\
        \hline
        full\_cpu\_bar\_color & Couleur de la barre de progression CPU (pleine) & DARK\_YELLOW \\
        \hline
        full\_gpu\_bar\_color & Couleur de la barre de progression GPU (pleine) & DARK\_GREEN \\
        \hline
        full\_memory\_bar\_color & Couleur de la barre de progression mémoire (pleine) & DARK\_MAGENTA \\
        \hline
        full\_disk\_bar\_color & Couleur de la barre de progression disque (pleine) & DARK\_BLUE \\
        \hline
        empty\_cpu\_bar\_color & Couleur de la barre de progression CPU (vide) & BRIGHT\_YELLOW \\
        \hline
        empty\_gpu\_bar\_color & Couleur de la barre de progression GPU (vide) & BRIGHT\_GREEN \\
        \hline
        empty\_memory\_bar\_color & Couleur de la barre de progression mémoire (vide) & BRIGHT\_MAGENTA \\
        \hline
        empty\_disk\_bar\_color & Couleur de la barre de progression disque (vide) & BRIGHT\_BLUE \\
        \hline
    \end{tabular}
    \label{tab:configable_color_var}
    \caption{Variables configurables avec une couleur}
\end{table}

\newpage
\subsection{Variables configurables avec un caractère UNICODE :}
\begin{table}[h!]
    \centering
    \renewcommand{\arraystretch}{1.5}
    \footnotesize
    \begin{tabular}{|>{\centering\arraybackslash}m{3cm}|>{\centering\arraybackslash}m{3cm}|>{\centering\arraybackslash}m{3cm}|}
        \hline
        \textbf{Nom de la variable} & \textbf{Signification} & \textbf{Valeur par défaut (caractère UNICODE)} \\
        \hline
        border\_char & Caractère représentant les bordures des fenêtres & unicode\_full\_block \\
        \hline
        full\_bar\_char & Caractère représentant la barre de progression pleine & unicode\_dark\_shade \\
        \hline
        empty\_bar\_char & Caractère représentant la barre de progression vide & unicode\_light\_shade \\
        \hline
    \end{tabular}
    \label{tab:configable_unicode_var}
    \caption{Variables configurables avec un caractère UNICODE}
\end{table}

\vspace{2em}
\subsection{Autres variables configurables :}
\begin{table}[h!]
    \centering
    \renewcommand{\arraystretch}{1.5}
    \footnotesize
    \begin{tabular}{|>{\centering\arraybackslash}m{3cm}|>{\centering\arraybackslash}m{1.6cm}|>{\centering\arraybackslash}m{3cm}|>{\centering\arraybackslash}m{1.6cm}|}
        \hline
        \textbf{Nom de la variable} & \textbf{Valeur} & \textbf{Signification} & \textbf{Valeur par défaut} \\
        \hline
        minimum\_lines\_width & number & Largeur minimale en lignes & 30 \\
        \hline
        minimum\_cols\_height & number & Hauteur minimale en colonnes & 70 \\
        \hline
        update\_log\_time & number & Fréquence de mise à jour du fichier de logs en secondes & 60 \\
        \hline
        overwrite\_log & true ou false & Indique si le fichier de logs \texttt{readconfigfile.txt} doit être écrasé & true \\
        \hline
    \end{tabular}
    \label{tab:configable_other_var}
    \caption{Autres variables configurables}
\end{table}

\newpage
\subsection{Couleurs disponibles :}
\begin{table}[h!]
    \centering
    \renewcommand{\arraystretch}{1.5}
    \footnotesize
    \begin{tabular}{|>{\centering\arraybackslash}m{4cm}|>{\centering\arraybackslash}m{3cm}|>{\centering\arraybackslash}m{3cm}|}
        \hline
        \textbf{Nom de la couleur} & \textbf{Code hexadécimal} & \textbf{Signification} \\
        \hline
        DARK\_BLACK & \cellcolor[HTML]{000000}\textcolor{white}{\#000000} & Noir foncé \\
        \hline
        DARK\_RED & \cellcolor[HTML]{800000}\textcolor{white}{\#800000} & Rouge foncé \\
        \hline
        DARK\_GREEN & \cellcolor[HTML]{008000}\textcolor{white}{\#008000} & Vert foncé \\
        \hline
        DARK\_YELLOW & \cellcolor[HTML]{808000}\textcolor{black}{\#808000} & Jaune foncé \\
        \hline
        DARK\_BLUE & \cellcolor[HTML]{000080}\textcolor{white}{\#000080} & Bleu foncé \\
        \hline
        DARK\_MAGENTA & \cellcolor[HTML]{800080}\textcolor{white}{\#800080} & Magenta foncé \\
        \hline
        DARK\_CYAN & \cellcolor[HTML]{008080}\textcolor{white}{\#008080} & Cyan foncé \\
        \hline
        DARK\_WHITE & \cellcolor[HTML]{C0C0C0}\textcolor{black}{\#C0C0C0} & Blanc/gris foncé \\
        \hline
        BRIGHT\_BLACK & \cellcolor[HTML]{808080}\textcolor{white}{\#808080} & Noir clair (gris foncé) \\
        \hline
        BRIGHT\_RED & \cellcolor[HTML]{FF0000}\textcolor{white}{\#FF0000} & Rouge clair \\
        \hline
        BRIGHT\_GREEN & \cellcolor[HTML]{00FF00}\textcolor{black}{\#00FF00} & Vert clair \\
        \hline
        BRIGHT\_YELLOW & \cellcolor[HTML]{FFFF00}\textcolor{black}{\#FFFF00} & Jaune clair \\
        \hline
        BRIGHT\_BLUE & \cellcolor[HTML]{0000FF}\textcolor{white}{\#0000FF} & Bleu clair \\
        \hline
        BRIGHT\_MAGENTA & \cellcolor[HTML]{FF00FF}\textcolor{black}{\#FF00FF} & Magenta clair \\
        \hline
        BRIGHT\_CYAN & \cellcolor[HTML]{00FFFF}\textcolor{black}{\#00FFFF} & Cyan clair \\
        \hline
        BRIGHT\_WHITE & \cellcolor[HTML]{FFFFFF}\textcolor{black}{\#FFFFFF} & Blanc/gris clair \\
        \hline
        BRIGHT\_BLACK\_BIS & \cellcolor[HTML]{555555}\textcolor{white}{\#555555} & Noir clair bis \\
        \hline
        BRIGHT\_RED\_BIS & \cellcolor[HTML]{FF5555}\textcolor{black}{\#FF5555} & Rouge clair bis \\
        \hline
        BRIGHT\_GREEN\_BIS & \cellcolor[HTML]{55FF55}\textcolor{black}{\#55FF55} & Vert clair bis \\
        \hline
        BRIGHT\_YELLOW\_BIS & \cellcolor[HTML]{FFFF55}\textcolor{black}{\#FFFF55} & Jaune clair bis \\
        \hline
        BRIGHT\_BLUE\_BIS & \cellcolor[HTML]{5555FF}\textcolor{white}{\#5555FF} & Bleu clair bis \\
        \hline
        BRIGHT\_MAGENTA\_BIS & \cellcolor[HTML]{FF55FF}\textcolor{black}{\#FF55FF} & Magenta clair bis \\
        \hline
        BRIGHT\_CYAN\_BIS & \cellcolor[HTML]{55FFFF}\textcolor{black}{\#55FFFF} & Cyan clair bis \\
        \hline
        BRIGHT\_WHITE\_BIS & \cellcolor[HTML]{E5E5E5}\textcolor{black}{\#E5E5E5} & Blanc/gris clair bis \\
        \hline
        DARK\_RED\_BIS & \cellcolor[HTML]{800000}\textcolor{white}{\#800000} & Fond rouge foncé \\
        \hline
        DARK\_GREEN\_BIS & \cellcolor[HTML]{008000}\textcolor{white}{\#008000} & Fond vert foncé \\
        \hline
        DARK\_YELLOW\_BIS & \cellcolor[HTML]{808000}\textcolor{black}{\#808000} & Fond jaune foncé \\
        \hline
        DARK\_BLUE\_BIS & \cellcolor[HTML]{000080}\textcolor{white}{\#000080} & Fond bleu foncé \\
        \hline
        DARK\_MAGENTA\_BIS & \cellcolor[HTML]{800080}\textcolor{white}{\#800080} & Fond magenta foncé \\
        \hline
        DARK\_CYAN\_BIS & \cellcolor[HTML]{008080}\textcolor{white}{\#008080} & Fond cyan foncé \\
        \hline
        DARK\_WHITE\_BIS & \cellcolor[HTML]{C0C0C0}\textcolor{black}{\#C0C0C0} & Fond blanc/gris foncé \\
        \hline
    \end{tabular}
    \caption{Couleurs disponibles}
    \label{tab:couleur}
\end{table}

\newpage
\subsection{Caractères UNICODE disponibles}
\begin{table}[h!]
    \centering
    \renewcommand{\arraystretch}{1.5} % Ajuste l'espacement vertical des lignes
    \footnotesize
    \begin{tabular}{|>{\centering\arraybackslash}m{4cm}|>{\centering\arraybackslash}m{2.3cm}|>{\centering\arraybackslash}m{3cm}|}
        \hline
        \textbf{Nom du caractère} & \textbf{Code Unicode} & \textbf{Signification} \\
        \hline
        unicode\_full\_block & \texttt{\textbackslash u2588} & Bloc complet \\
        \hline
        unicode\_upper\_half\_block & \texttt{\textbackslash u2580} & Demi-bloc supérieur \\
        \hline
        unicode\_lower\_half\_block & \texttt{\textbackslash u2584} & Demi-bloc inférieur \\
        \hline
        unicode\_left\_half\_block & \texttt{\textbackslash u258C} & Demi-bloc gauche \\
        \hline
        unicode\_right\_half\_block & \texttt{\textbackslash u2590} & Demi-bloc droit \\
        \hline
        unicode\_light\_shade & \texttt{\textbackslash u2591} & Ombrage léger \\
        \hline
        unicode\_medium\_shade & \texttt{\textbackslash u2592} & Ombrage moyen \\
        \hline
        unicode\_dark\_shade & \texttt{\textbackslash u2593} & Ombrage foncé \\
        \hline
        unicode\_white\_square & \texttt{\textbackslash u25A1} & Carré blanc \\
        \hline
        unicode\_black\_circle & \texttt{\textbackslash u25CF} & Cercle noir \\
        \hline
        unicode\_white\_circle & \texttt{\textbackslash u25CB} & Cercle blanc \\
        \hline
        unicode\_black\_diamond & \texttt{\textbackslash u25C6} & Losange noir \\
        \hline
        unicode\_white\_diamond & \texttt{\textbackslash u25C7} & Losange blanc \\
        \hline
        unicode\_black\_star & \texttt{\textbackslash u2605} & Étoile noire \\
        \hline
        unicode\_white\_star & \texttt{\textbackslash u2606} & Étoile blanche \\
        \hline
    \end{tabular}
    \label{tab:unicode}
    \caption{Caractères UNICODE disponibles}
\end{table}

\end{document}
